\documentclass[sigconf]{acmart}

%%
%% \BibTeX command to typeset BibTeX logo in the docs
\AtBeginDocument{%
  \providecommand\BibTeX{{%
    \normalfont B\kern-0.5em{\scshape i\kern-0.25em b}\kern-0.8em\TeX}}}

%% Rights management information.  This information is sent to you
%% when you complete the rights form.  These commands have SAMPLE
%% values in them; it is your responsibility as an author to replace
%% the commands and values with those provided to you when you
%% complete the rights form.
\setcopyright{acmcopyright}
\copyrightyear{2018}
\acmYear{2018}
\acmDOI{10.1145/1122445.1122456}

%% These commands are for a PROCEEDINGS abstract or paper.
\acmConference[Woodstock '18]{Woodstock '18: ACM Symposium on Neural
  Gaze Detection}{June 03--05, 2018}{Woodstock, NY}
\acmBooktitle{Woodstock '18: ACM Symposium on Neural Gaze Detection,
  June 03--05, 2018, Woodstock, NY}
\acmPrice{15.00}
\acmISBN{978-1-4503-XXXX-X/18/06}


%%
%% Submission ID.
%% Use this when submitting an article to a sponsored event. You'll
%% receive a unique submission ID from the organizers
%% of the event, and this ID should be used as the parameter to this command.
%%\acmSubmissionID{123-A56-BU3}

%%
%% The majority of ACM publications use numbered citations and
%% references.  The command \citestyle{authoryear} switches to the
%% "author year" style.
%%
%% If you are preparing content for an event
%% sponsored by ACM SIGGRAPH, you must use the "author year" style of
%% citations and references.
%% Uncommenting
%% the next command will enable that style.
%%\citestyle{acmauthoryear}

%%
%% end of the preamble, start of the body of the document source.
\begin{document}

%%
%% The "title" command has an optional parameter,
%% allowing the author to define a "short title" to be used in page headers.
\title{Identification of Most Influential Spreaders in Twitter Social Network using Modified K Core Decomposition in Distributed Environment}
\begin{comment}
%%
%% The "author" command and its associated commands are used to define
%% the authors and their affiliations.
%% Of note is the shared affiliation of the first two authors, and the
%% "authornote" and "authornotemark" commands
%% used to denote shared contribution to the research.
\author{TM Tariq Adnan}
%\authornote{Both authors contributed equally to this research.}
%\email{tmadnan10@gmail.com}
%\orcid{1234-5678-9012}
%\author{G.K.M. Tobin}
%\authornotemark[1]
%\email{webmaster@marysville-ohio.com}
%\affiliation{%
%  \institution{Institute for Clarity in %Documentation}
%  \streetaddress{P.O. Box 1212}
%  \city{Dublin}
%  \state{Ohio}
%  \postcode{43017-6221}
%}



\author{Md Saiful Islam}
%\email{saiful.11722@gmail.com}
\author{Tarikul Islam Papon}
\author{Shourav Nath}
\author{Touhidul Islam}
\author{Muhammad Abdullah Adnan}
\affiliation{%
  \institution{Bangladesh University of Engineering and Technology (BUET)}
  %\streetaddress{1 Th{\o}rv{\"a}ld Circle}
  \city{Dhaka}
  \country{Bangladesh}}



\affiliation{%
  \institution{Inria Paris-Rocquencourt}
  \city{Rocquencourt}
  \country{France}
}

\author{Aparna Patel}
\affiliation{%
 \institution{Rajiv Gandhi University}
 \streetaddress{Rono-Hills}
 \city{Doimukh}
 \state{Arunachal Pradesh}
 \country{India}}

\author{Huifen Chan}
\affiliation{%
  \institution{Tsinghua University}
  \streetaddress{30 Shuangqing Rd}
  \city{Haidian Qu}
  \state{Beijing Shi}
  \country{China}}

\author{Charles Palmer}
\affiliation{%
  \institution{Palmer Research Laboratories}
  \streetaddress{8600 Datapoint Drive}
  \city{San Antonio}
  \state{Texas}
  \postcode{78229}}
\email{cpalmer@prl.com}

\author{John Smith}
\affiliation{\institution{The Th{\o}rv{\"a}ld Group}}
\email{jsmith@affiliation.org}

\author{Julius P. Kumquat}
\affiliation{\institution{The Kumquat Consortium}}
\email{jpkumquat@consortium.net}

%%
%% By default, the full list of authors will be used in the page
%% headers. Often, this list is too long, and will overlap
%% other information printed in the page headers. This command allows
%% the author to define a more concise list
%% of authors' names for this purpose.
\renewcommand{\shortauthors}{Trovato and Tobin, et al.}

\end{comment}
%%
%% The abstract is a short summary of the work to be presented in the
%% article.
\begin{abstract}
  A clear and well-documented \LaTeX\ document is presented as an
  article formatted for publication by ACM in a conference proceedings
  or journal publication. Based on the ``acmart'' document class, this
  article presents and explains many of the common variations, as well
  as many of the formatting elements an author may use in the
  preparation of the documentation of their work.
\end{abstract}

%%
%% The code below is generated by the tool at http://dl.acm.org/ccs.cfm.
%% Please copy and paste the code instead of the example below.
%%
\begin{CCSXML}
<ccs2012>
 <concept>
  <concept_id>10010520.10010553.10010562</concept_id>
  <concept_desc>Computer systems organization~Embedded systems</concept_desc>
  <concept_significance>500</concept_significance>
 </concept>
 <concept>
  <concept_id>10010520.10010575.10010755</concept_id>
  <concept_desc>Computer systems organization~Redundancy</concept_desc>
  <concept_significance>300</concept_significance>
 </concept>
 <concept>
  <concept_id>10010520.10010553.10010554</concept_id>
  <concept_desc>Computer systems organization~Robotics</concept_desc>
  <concept_significance>100</concept_significance>
 </concept>
 <concept>
  <concept_id>10003033.10003083.10003095</concept_id>
  <concept_desc>Networks~Network reliability</concept_desc>
  <concept_significance>100</concept_significance>
 </concept>
</ccs2012>
\end{CCSXML}

\ccsdesc[500]{Computer systems organization~Embedded systems}
\ccsdesc[300]{Computer systems organization~Redundancy}
\ccsdesc{Computer systems organization~Robotics}
\ccsdesc[100]{Networks~Network reliability}

%%
%% Keywords. The author(s) should pick words that accurately describe
%% the work being presented. Separate the keywords with commas.
\keywords{datasets, neural networks, gaze detection, text tagging}

%% A "teaser" image appears between the author and affiliation
%% information and the body of the document, and typically spans the
%% page.
%%
%% This command processes the author and affiliation and title
%% information and builds the first part of the formatted document.
\maketitle

\section{Introduction}

\section{Technical Background}

\subsection{Centrality Measurement}
\subsubsection{Degree Centrality}


\subsubsection{Closeness Centrality}

\subsubsection{Betweenness Centrality}

\subsubsection{Eigen Vector Centrality}


\section{Related Work}


\section{Experimental Setup}

\subsection{Datasets}


\section{Evaluation}



\section{Evaluation Metrics}
\subsection{Rank Correlation Coefficient}

In general, correlation analyses are bi-variate analyses that measure the strength of association between two variables and the direction of the relationship. In terms of the strength of relationship, the value of the correlation coefficient varies between $+1$ and $-1$. A value of $\pm 1$ indicates that there exists a perfect degree of association between the comparing two variables. As the correlation coefficient value goes towards $0$, this relationship between the two variables gets weaker. The $\pm$ sings of the coefficient indicated he direction of the relationship; a $+$ sign indicates a positive relationship while a $—$ sign indicates a negative relationship.

%In general, correlation analyses measure the strength of the relationship between two variables. They assess statistical associations based on the ranks of the data. Correlation coefficients take the values between minus one and plus one. The positive correlation signifies that the ranks of both the variables are increasing.  On the other hand, the negative correlation signifies that as the rank of one variable is increased, the rank of the other variable is decreased.

%Ranking data is carried out on the variables that are separately put in order and are numbered.

In Section \ref{?}, we measure the correlation of the ranked list of users based on their spreadability generated by our modified K core decomposition against the rankings generated by other methods. We use two non-parametric rank correlations: Kendall’s tau and Spearman’s rank correlation coefficient.

\subsubsection{Kendall Tau Correlation Co-efficient}

The Kendall tau rank correlation co-efficient is used to test the similarities in the ordering of data when it is ranked by quantities. While other types of correlation coefficients use the observations as the basis of the correlation, Kendall’s correlation coefficient uses pairs of observations and determines the strength of association based on the patter on concordance and discordance between the pairs. Assume that $L_1$ and $L_2$ are the two rankings that are to be compared. Then Kendall analysis takes the following two properties into consideration:

\begin{itemize}
    \item \textbf{Concordant}: Any pair of items $(x_1,y_1)$ in $L_1$ and $(x_2,y_2)$ in $L_2$ are considered as concordant if and only if they meet one of the following two conditions:
    
    \begin{itemize}
        \item ($rank\_in\_L_1(x_1) > rank\_in\_L_2(x_2)$ and 
        
        $rank\_in\_L_1(y_1) > rank\_in\_L_2(y_2)$)
        
        \item ($rank\_in\_L_2(x_2) > rank\_in\_L_1(x_1)$ and 
        
        $rank\_in\_L_2(y_2) > rank\_in\_L_1(y_1)$)
    \end{itemize}
    
    \item \textbf{Discordant}: Any pair of items $(x_1,y_1)$ in $L_1$ and $(x_2,y_2)$ in $L_2$ are considered as discordant if and only if they meet one of the following two conditions:
    
    \begin{itemize}
        \item ($rank\_in\_L_1(x_1) > rank\_in\_L_2(x_2)$ and 
        
        $rank\_in\_L_1(y_1) < rank\_in\_L_2(y_2)$)
        
        \item ($rank\_in\_L_2(x_2) > rank\_in\_L_1(x_1)$ and 
        
        $rank\_in\_L_2(y_2) < rank\_in\_L_1(y_1)$)
    \end{itemize}
\end{itemize}


 %The Kendall’s tau coefficient considers a set of joint observations from two random variables X and Y. Any pair of observation (xi, yi) and (xj, yj) are said to be concordant if the ranks for both elements agree: that is, if both xi >xj and yi >yj or if both xi <xj and yi <yj. 
 
 %They are said to be discordant if xi >xj and yi <yj or if xi <xj and yi >yj. It is defined as follows:
 
Kendall Tau correlation co-efficient is denoted by $\tau$. If $L_1$ and $L_2$ are two different rankings with $n$ similar elements, $N(C)$ and $N(D)$ represent the number of concordant and discordant pair respectively, then $\tau$ can be calculated using the following equation:
\begin{equation}
    \tau(L_1, L_2) = \dfrac{N(C)-N(D)}{\frac{1}{2}n(n-1)}
\end{equation}

\subsubsection{Spearman's Rank Correlation Co-efficient}
Spearman’s Rank correlation coefficient is a technique which can be used to summarise the strength and direction (negative or positive) of a relationship between two variables.
The result will always be between 1 and minus 1. Assume that $L_1$ and $L_2$ are two ranking of same $n$ elements. For any element $x$, if the ranking of $x$ in $L_1$ and $L_2$ are $rank\_in\_L_1(x)$

This technique initially finds the difference in the ranks ($d$) : This is the difference between the ranks of the two values on each row of the table. The rank of the second value (price) is subtracted from the rank of the first (distance from the museum).
Square the differences (d²) To remove negative values and then sum them (d²).
 
\subsubsection{Calculating Statistical Significance using Co-efficient Values}



\subsection{Infection Scale on SIR Model}



\end{document}
